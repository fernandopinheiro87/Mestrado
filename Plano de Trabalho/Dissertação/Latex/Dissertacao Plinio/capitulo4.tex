\chapter{Trabalhos Relacionados}
%oportunidade de usar o trabalho de outros autores para motivar o seu próprio estudo

Inicialmente, é importante destacar que não há o objetivo de realizar um mapeamento sistemático da literatura neste trabalho. O intuito deste capítulo é apresentar trabalhos com temática diretamente relacionada a esta pesquisa. Os resultados das buscas por trabalhos relacionados foram classificados por relevância e selecionados informalmente, priorizando trabalhos recentes e com abordagens correlatas ao problema de pesquisa aqui apresentado.

Para buscar trabalhos relacionados ao tema foi utilizado o engenho de busca contido no Portal de Periódicos da CAPES, além da base de teses e dissertações. A Tabela \ref{tabela:stringsIngles} apresenta as \textit{strings} utilizadas nas buscas em língua inglesa e a Tabela \ref{tabela:stringsPortugues} apresenta as \textit{strings} utilizadas nas buscas em língua portuguesa. As buscas limitaram-se a trabalhos dos últimos seis anos, de 2011 a 2016.

% http://www.tablesgenerator.com/
% Please add the following required packages to your document preamble:
% \usepackage[table,xcdraw]{xcolor}
% If you use beamer only pass "xcolor=table" option, i.e. \documentclass[xcolor=table]{beamer}
\begin{table}[h!]
\centering
\begin{tabular}{|l|c|c|}
\hline
\rowcolor[HTML]{9B9B9B} 
\multicolumn{1}{|c|}{\cellcolor[HTML]{9B9B9B}{\color[HTML]{000000} \textit{\textbf{String}}}}                                    & {\color[HTML]{000000} \textbf{\begin{tabular}[c]{@{}c@{}}Trabalhos \\ retornados\end{tabular}}} & {\color[HTML]{000000} \textbf{\begin{tabular}[c]{@{}c@{}}Trabalhos\\ selecionados\end{tabular}}} \\ \hline
"IT Planning" (título)                                                                                                           & 7                                                                                               & 0                                                                                                \\ \hline
"Information Techonology Planning" (título)                                                                                      & 0                                                                                               & 0                                                                                                \\ \hline
\begin{tabular}[c]{@{}l@{}}"IT Planning" (título) AND \\ "government" (documento todo)\end{tabular}                              & 3                                                                                               & 0                                                                                                \\ \hline
\begin{tabular}[c]{@{}l@{}}"IT Planning" (título) AND \\ "Public Sector" (documento todo)\end{tabular}                           & 1                                                                                               & 0                                                                                                \\ \hline
"Strategic Information Systems Planning" (título)                                                                                & 15                                                                                              & 0                                                                                                \\ \hline
\begin{tabular}[c]{@{}l@{}}"Strategic Information Systems Planning" (título) \\ AND "government" (documento todo)\end{tabular}   & 5                                                                                               & 0                                                                                                \\ \hline
\begin{tabular}[c]{@{}l@{}}"Strategic Information Systems Planning" (título)\\ AND "public sector" (documento todo)\end{tabular} & 2                                                                                               & 0                                                                                                \\ \hline
\end{tabular}
\caption{\textit{Strings} de busca em periódicos na língua inglesa }
\label{tabela:stringsIngles}
\end{table}


% Please add the following required packages to your document preamble:
% \usepackage[table,xcdraw]{xcolor}
% If you use beamer only pass "xcolor=table" option, i.e. \documentclass[xcolor=table]{beamer}
\begin{table}[h!]
\centering
\begin{tabular}{|l|c|c|}
\hline
\rowcolor[HTML]{9B9B9B} 
\multicolumn{1}{|c|}{\cellcolor[HTML]{9B9B9B}{\color[HTML]{000000} \textit{\textbf{String}}}} & {\color[HTML]{000000} \textbf{\begin{tabular}[c]{@{}c@{}}Trabalhos \\ retornados\end{tabular}}} & {\color[HTML]{000000} \textbf{\begin{tabular}[c]{@{}c@{}}Trabalhos\\ selecionados\end{tabular}}} \\ \hline
"Planejamento de TI" (título) & 0 & 0 \\ \hline
"Planejamento de tecnologia da informação" (título) & 4 & 0 \\ \hline
"PDTI" (título) & 1 & 0 \\ \hline
"Plano diretor de TI" (título) & 1 & 0 \\ \hline
"Plano de TI" (título) & 0 & 0 \\ \hline
"Planos de TI" (título) & 1 & 1 \\ \hline
"Planejamento estratégico de TI" (título) & 13 & 0 \\ \hline
"Planejamento estratégico de\\ Tecnologia da Informação" (título) & 5 & 2 \\ \hline
\end{tabular}
\caption{\textit{Strings} de busca em trabalhos na língua portuguesa}
\label{tabela:stringsPortugues}
\end{table}

As buscas por trabalhos relacionados revelaram que a maior parte das produções acadêmicas sobre planejamento de TI se referem ao setor privado. Nos trabalhos mais recentes, das poucas publicações que abordam o planejamento de TI no setor público brasileiro a grande maioria são teses ou dissertações. O contrário ocorre com as publicações em língua inglesa, onde percebe-se que as publicações que abordam planejamento de TI no setor público são artigos de periódicos e conferências.

Na seleção dos trabalhos, o principal critério de seleção consistiu na relação entre o planejamento de TI e setor público. Posteriormente, foram filtrados apenas trabalhos focados em planejamentos de TI de organizações públicas brasileiras cujo escopo contribuísse para a presente pesquisa. Os três trabalhos relacionados são dissertações de mestrado: \citeonline{paula:12}; \citeonline{barros:13}; \citeonline{prando:15}.

\section{Fatores Condicionantes Relacionados ao Planejamento de TI}
% do problema de pesquisa
A caracterização do problema na dissertação de \citeonline{paula:12} evidencia que naquela ocasião já era visível que, apesar da obrigatoriedade de realizar o planejamento de TI, os órgãos públicos brasileiros encontravam dificuldades para realizar tal atividade. Diante disso, \citeonline{paula:12} pesquisou sobre modelos de planejamento de TI aderentes ao setor público. Ao investigar as práticas de formulação e implantação do planejamento de TI em órgãos federais, \citeonline{barros:13} identificou fatores condicionantes que se relacionam com a formulação e implantação de planos de TI. No trabalho de \citeonline{prando:15}, um dos objetivos consiste em explorar as lacunas e conflitos presentes no processo de planejamento de TI de uma instituição pública federal em um cenário de expansão.

%dos fatores que dificultam o planejamento

\citeonline{paula:12} utilizou-se da literatura científica para levantar a hipótese de sua pesquisa. Nesta abordagem, a falta de um modelo de formulação de planejamento estratégico de TI, voltado para instituições universitárias federais, seria o fator restritivo para que estas instituições atendessem satisfatoriamente o planejamento de TI. 

Para identificar os fatores condicionantes que influem na elaboração e execução dos planos de TI de órgãos públicos federais, \citeonline{barros:13} entrevistou dirigentes de TI codificando seus depoimentos de acordo com as categorias do modelo teórico de \citeonline{brown:04}. Desta forma, o autor organizou as percepções dos entrevistados a respeito do planejamento de TI sob a perspectiva das dez categorias do trabalho de \citeonline{brown:04}, que por sua vez, foram mapeadas a partir da literatura científica.

No trabalho de \citeonline{prando:15}, a pesquisa baseou-se em cinco proposições acerca das dificuldades relacionadas à elaboração do PDTI em uma determinada instituição. As proposições foram expostas à validação através de pesquisa documental, entrevistas e estudo de caso.

Um dos objetivos deste trabalho é identificar os fatores que dificultam ou impedem a elaboração do planejamento de TI em órgãos públicos federais. Desta forma, com o intuito de realizar comparações após os resultados da pesquisa, a Tabela \ref{tabela:fatoresTrabRelacionados} foi elaborada compilando os principais fatores que dificultam o planejamento de TI, sob a perspectiva de cada trabalho relacionado neste capítulo.

% Please add the following required packages to your document preamble:
% \usepackage{graphicx}
% \usepackage[table,xcdraw]{xcolor}
% If you use beamer only pass "xcolor=table" option, i.e. \documentclass[xcolor=table]{beamer}
\begin{table}[h!]
\centering
\resizebox{\textwidth}{!}{%
\begin{tabular}{|l|l|l|}
\hline
\rowcolor[HTML]{9B9B9B} 
\multicolumn{1}{|c|}{\cellcolor[HTML]{9B9B9B}{\color[HTML]{000000} Paula (2012)}}                                                      & \multicolumn{1}{c|}{\cellcolor[HTML]{9B9B9B}{\color[HTML]{000000} Barros (2013)}}                                                                                                                                                                                                                                                                                                                                                                                                                                                                                                                                                                                                                                                              & \multicolumn{1}{c|}{\cellcolor[HTML]{9B9B9B}{\color[HTML]{000000} Prando (2015)}}                                                                                                                                                                                                                                       \\ \hline
\begin{tabular}[c]{@{}l@{}}- Dificuldade em aplicar modelos de \\ planejamento de TI no cenário de instituição\\ pública.\end{tabular} & \begin{tabular}[c]{@{}l@{}}- Influência (negativa) da alta gestão;\\ - Mudanças no cenário político impedem o planejamento\\  a longo prazo;\\ - Contingenciamento orçamentário;\\ - Dificuldade em consolidar as informações de todas as \\ áreas em instituições com grande número de unidades;\\ - Rotatividade de dirigentes e descontinuidade \\ administrativa;\\ - Falta de governança corporativa;\\ - Falta de cultura de planejamento e de pensamento \\ estratégico;\\ - TI mal posicionada na hierarquia organizacional;\\ - Falta pessoal de TI;\\ - Falta qualificação do quadro de pessoal;\\ - Dificuldades para conseguir a participação das áreas\\ finalísticas;\\ - Dificuldades na priorização das demandas.\end{tabular} & \begin{tabular}[c]{@{}l@{}}- Expansão acelerada da instituição;\\ - Parte dos coordenadores de TI desconhecem\\ o PDTI;\\ - Alinhamento estratégico fraco ou inexistente;\\ - Coordenadores de TI há pouco tempo no cargo;\\ - Dificuldade em manter o comprometimento\\ dos participantes do comitê de TI.\end{tabular} \\ \hline
\end{tabular}%
}
\caption{Fatores que dificultam o planejamento de TI, segundo trabalhos relacionados.}
\label{tabela:fatoresTrabRelacionados}
\end{table}

\section{Métodos Utilizados}
Os métodos empregados em cada trabalho relacionado neste capítulo também contribuem para a presente pesquisa. O intuito desta seção é destacar a metodologia de cada pesquisa para poder comparar os trabalhos e interpretar os resultados de forma ponderada. Desta maneira, não é objeto desta seção descrever os métodos, mas espera-se expor as características relevantes na influência dos resultados e suas análises.

\citeonline{paula:12} utilizou o método pesquisa-ação, que pode ser definida como ``toda tentativa continuada, sistemática e empiricamente fundamentada de aprimirar a prática'' \cite{tripp:05}. Neste cenário, a prática envolvida consiste na formulação do planejamento estratégico de TI de uma determinada instituição. A pesquisa-ação desenvolveu-se em três etapas: planejar, agir e refletir.

\begin{citacao}
Na etapa planejar, foi realizada a avaliação dos fatores que influenciam o planejamento estratégico de TI na UNIRIO e a análise dos modelos de planejamento estratégico de TI existentes na literatura. Na etapa Agir foi aplicado o Modelo de Planejamento Estratégico de TI desenvolvido pelo SISP, tendo como resultado o PDTIC da UNIRIO. Na etapa Refletir, foram apresentadas as reflexões sobre os efeitos decorrentes da aplicação da modelo do SISP e a identificação de possíveis melhorias para futuros ciclos de planejamento \cite{paula:12}.
\end{citacao}

No método utilizado por \citeonline{paula:12}, destaca-se a fase ``planejar'', na qual avaliou-se os fatores que influenciam o planejamento de TI da instituição. A autora confirmou seu problema de pesquisa nesta etapa ao traçar o cenário da instituição com relação ao planejamento de TI. Para isto, analisou o histórico da instituição; buscou a origem do problema de planejamento de TI estabelecendo contato com os participantes e interessados. Há de se destacar que a pesquisadora interferiu no ambiente da pesquisa por ser membro do comitê de TI da instituição pesquisada. Porém, o método utilizado permite tal interferência.

A pesquisa de \citeonline{barros:13} é de natureza qualitativa, aplicando entrevistas semiestruturadas à alguns dirigentes de TI de organizações públicas federais. Os dados coletados são codificados de acordo com as categorias pré-concebidas no modelo teórico de \citeonline{brown:04}, ou seja, possui traços de \textit{Grounded Theory}. Desta forma, o pesquisador não apresenta uma teoria fundamentada nos dados, mas uma classificação de códigos de acordo com um modelo teórico fundamentado na literatura.

\citeonline{prando:15} também optou pela abordagem qualitativa, porém utilizando os métodos de estudo de caso, entrevistas em profundidade e pesquisa documental. Tal metodologia permitiu ao pesquisador traçar o retrato fiel da instituição pesquisada e propor recomendações direcionadas aos problemas levantados. Porém, esta abordagem restringe a generalização do problema para outras instituições, uma vez que os fatores condicionantes levantados são de natureza específica da instituição em questão.

\section{Soluções Propostas}
O trabalho de \citeonline{paula:12} permitiu a adaptação de um modelo de planejamento de TI existente, gerando um modelo exclusivo voltado para atender à instituição pesquisada. Desta forma, a pesquisadora buscou eliminar as lacunas e dificuldades que a instituição enfrentava no processo de elaboração do PDTI. Contudo, o contexto de outras instituições pode ser diferente e, com isto, o modelo proposto pode não ser aplicável.

A abordagem do trabalho de \citeonline{barros:13} não tem como objetivo propor soluções aos problemas relacionados ao planejamento de TI. A proposta é explorar a perspectiva dos dirigentes de TI e expor os fatores que influenciam a formulação e execução dos planos de TI. Os relatos dos dirigentes são ricos e o pesquisador os apresenta de forma a mostrar contrapontos e justificativas dos entrevistados.

A pesquisa de \citeonline{prando:15} oferece um conjunto de recomendações com o intuito de reduzir os problemas levantados. O pesquisador não especifica a fonte das recomendações, desta forma infere-se que as sugestões foram baseadas no conhecimento do próprio pesquisador. Por se tratar de uma pesquisa com foco específico em problemas de planejamento de TI de uma determinada instituição, fica dificultada a generalização das recomendações propostas e, consequentemente, a aplicação em outras instituições.

\section{Considerações}
Os três trabalhos relacionados corroboram com a motivação da presente pesquisa fornecendo abordagens de identificação das dificuldades e deficiências no planejamento de TI de organizações públicas e propondo ações de melhoria para o cenário pesquisado.

A proposta da presente pesquisa, ao utilizar \textit{Grounded Theory}, não utiliza-se de premissas ou hipóteses assim como nos trabalhos de \citeonline{paula:12} e \citeonline{prando:15}. Ao contrário, as proposições são levantadas na análise dos dados coletados e refutadas ou confirmadas também nos dados no desenvolver da pesquisa.

É interessante observar que \citeonline{barros:13} utilizou \textit{Grounded Theory} de forma indireta em sua pesquisa. Isto pode ser afirmado pois o autor codifica os dados utilizando categorias mapeadas anteriormente por \citeonline{brown:04}, que utilizaram GT para formular tal modelo. A criação das categorias da presente pesquisa foi feita de forma diferente, pois foram mapeadas a partir dos dados coletados exclusivamente para esta pesquisa.

Os três trabalhos apresentados nesta pesquisa comungam de um problema prático presente nas instituições públicas brasileiras: as dificuldades em se planejar de forma eficaz as ações de TI de uma organização pública. As pesquisas aqui apresentadas apontam que a busca pela identificação dos fatores que levam a este problema é uma tarefa complexa e subjetiva. Porém, ainda se faz necessário compreender de forma abrangente os fatores limitadores do planejamento de TI e expor as relações causais entre estes limitadores. Espera-se sanar esta necessidade com uma teoria abrangente, ou seja, genérica o suficiente para aplicação em diversas organizações; e com fundamentação empírica para que a teoria reflita diretamente a prática, permitindo a proposição de soluções eficazes.

%Porém, compreender a origem do problema é um passo necessário para propor as soluções.