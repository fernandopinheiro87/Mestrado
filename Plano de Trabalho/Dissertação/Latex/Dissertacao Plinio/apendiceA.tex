\chapter{Questionário de Coleta de Dados} 
\label{apendice:a_quest_coleta}

\textit{As perguntas iniciais se referem aos dados do respondente:}
\\
\\I - Em qual instituição você trabalha?
\\II - Qual o seu cargo efetivo?
\\III - Possui Função Gratificada ou Cargo de Direção?
\\IV - Tempo na instituição atual.
\\V - Formação acadêmica: 
(a) Ensino Médio; (b) Superior; (c) Pós-graduação Lato Sensu; (d) Mestrado; (e) Doutorado; (f) Outra.

\textit{A pergunta a seguir delimita os grupos de respondentes desviando o fluxo de perguntas para questionamentos específicos de cada grupo:}
\\
\\1 - Sua instituição possui um PDTI? (Sim ou Não)

\textit{Para melhor organização deste apêndice, as perguntas a seguir foram numeradas de acordo com o grupo de respondentes. As perguntas iniciadas pelo número 1 correspondem ao grupo de instituições que não possuíam; as questões iniciadas com o número 2 correspondem ao grupo de instituições que possuem PDTI.}
\\
\\1.1 - Quais os impedimentos ou dificuldades que a gestão de TI da sua instituição encontrou que justifique a falta de um PDTI?
\\1.2 - Na sua opinião, qual seria a importância de um PDTI para o setor de TI da sua instituição?
\\1.3 - Na sua opinião, qual seria a importância de um PDTI para a instituição como um todo?
\\
\\2.1 - Você participou da elaboração ou de alguma revisão do PDTI da sua instituição? (Sim ou Não)
\\2.2 - O PDTI da sua instituição seguiu o Guia do SISP? (Sim ou Não)
\\2.3 - Identificar as necessidades de informação é uma das atividades no levantamento de necessidades para compor o PDTI. Assinale o grau de dificuldade que você ou sua equipe encontrou para identificar as necessidades de informação da instituição. (Escala de 0 a 10, onde 0 indica "nenhuma dificuldade" e 10 "muita dificuldade").
\\2.4 - Identificar as necessidades de contratação de serviços de TI é uma das atividades no levantamento de necessidades para compor o PDTI. Assinale o grau de dificuldade que você ou sua equipe encontrou para identificar as necessidades de contratação de serviços de TI para a instituição. (Escala de 0 a 10, onde 0 indica "nenhuma dificuldade" e 10 "muita dificuldade").
\\2.5 - Identificar as necessidades de infraestrutura de TI é uma das atividades no levantamento de necessidades para compor o PDTI. Assinale o grau de dificuldade que você ou sua equipe encontrou para identificar as necessidades de infraestrutura de TI para a instituição. (Escala de 0 a 10, onde 0 indica "nenhuma dificuldade" e 10 "muita dificuldade").
\\2.6 - Identificar as necessidades de pessoal de TI é uma das atividades no levantamento de necessidades para compor o PDTI. Assinale o grau de dificuldade que você ou sua equipe encontrou para identificar as necessidades de pessoal de TI da instituição. (Escala de 0 a 10, onde 0 indica "nenhuma dificuldade" e 10 "muita dificuldade").
\\2.7 - A respeito do diagnóstico das necessidades de TI da instituição, fale sobre as dificuldades encontradas durante o processo de levantamento de necessidades na sua instituição.
\\2.8 - Avalie a afirmação: No PDTI da sua instituição existe uma ordem de prioridade das necessidades inventariadas. (Escala de 1 a 5, onde 1 indica "discordo totalmente" e 5 "concordo totalmente").
\\2.9 - Qual o grau de dificuldade que você ou sua equipe encontrou para definir critérios e priorizar as necessidades de TI. (Escala de 0 a 10, onde 0 indica "nenhuma dificuldade" e 10 "muita dificuldade").
\\2.10 - Fale sobre as dificuldades encontradas na elaboração de critérios e priorização das necessidades da sua instituição.
\\2.11 - O item "Ampliar e atualizar o parque computacional nos Laboratórios de Informática e espaços de ensino" foi retirado de um PDTI. Na sua opinião, este item é um(a): 
(a) Necessidade de TI; (b) Ação; (c) Meta; (d) Nenhuma das alternativas.
\\2.12 - O item "Adequar a infraestrutura de 100\% dos laboratórios de informática e espaços de ensino até 2018" foi retirado de um PDTI. Na sua opinião, este item é um(a):
(a) Necessidade de TI; (b) Ação; (c) Meta; (d) Nenhuma das alternativas.
\\2.13 - O item "Formalizar e implantar processo de aquisição de equipamentos de TI" foi retirado de um PDTI. Na sua opinião, este item é um(a):
(a) Necessidade de TI; (b) Ação; (c) Meta; (d) Nenhuma das alternativas.
\\2.14 - Avalie a afirmação: No PDTI da sua instituição, as ações (atividades ou projetos) possuem metas a serem cumpridas. (Escala de 1 a 5, onde 1 indica "discordo totalmente" e 5 "concordo totalmente").
\\2.15 - Fale sobre as dificuldades encontradas na etapa de definição de metas e ações. (Escala de 1 a 5, onde 1 indica "discordo totalmente" e 5 "concordo totalmente").
\\2.16 - Avalie a afirmação: A instituição define no PDTI as diretrizes para gestão dos riscos de TI. (Escala de 1 a 5, onde 1 indica "discordo totalmente" e 5 "concordo totalmente").
\\2.17 - Fale sobre as dificuldades encontradas na etapa de planejar o gerenciamento de riscos.
\\2.18 - Avalie a afirmação: A instituição define no PDTI diretrizes para garantir o desenvolvimento de competências, contratação e a retenção de recursos humanos de TI. (Escala de 1 a 5, onde 1 indica "discordo totalmente" e 5 "concordo totalmente").
\\2.19 - Avalie a afirmação: No PDTI da sua instituição, os objetivos da TI estão alinhados com os objetivos estratégicos da instituição. (Escala de 1 a 5, onde 1 indica "discordo totalmente" e 5 "concordo totalmente").
\\2.20 - Na sua opinião, qual a importância do PDTI para o setor de TI da sua instituição?
\\2.21 - Na sua opinião, qual a importância do PDTI para a instituição como um todo?
\\2.22 - O PDTI é apenas uma das ferramentas existentes para se planejar ações de TI. Sua instituição possui algum outro plano ou ferramenta de planejamento de TI? Qual?
\\2.23 - Comente sobre dificuldades que a equipe que participou da elaboração ou revisão do PDTI enfrentou ao realizar o levantamento de necessidades de TI. (Pergunta destinada aos respondentes que responderam "não" na questão 2.1).
\\2.24 - Comente sobre dificuldades que a equipe que participou da elaboração ou revisão do PDTI enfrentou para definir critérios e priorizar as necessidades de TI. (Pergunta destinada aos respondentes que responderam "não" na questão 2.1).
\\2.25 - Comente sobre qualquer outra dificuldade que tenha chegado ao seu conhecimento que a equipe que participou da elaboração ou revisão do PDTI tenha enfrentado. (Pergunta destinada aos respondentes que responderam "não" na questão 2.1).

