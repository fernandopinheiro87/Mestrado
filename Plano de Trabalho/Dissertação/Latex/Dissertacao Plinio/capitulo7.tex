\chapter{Considerações Finais}

%PROBLEMA: "Dados do TCU revelam que muitos entes da Administração Pública Federal não cumprem a determinação de realizar o planejamento da TI. Além disso, nos planos de TI dos órgãos que o fazem, são encontradas deficiências que comprometem sua eficácia. Em suma, a falta de planejamento de TI evidencia o baixo nível de maturidade em governança de TI nestas instituições.". 

%QUESTÃO: "apesar da obrigatoriedade e dos conhecidos benefícios, o planejamento de TI não é realizado satisfatoriamente nos órgãos públicos federais. A atividade de planejamento envolve aspectos técnicos e sociais, diante disso, quais os fatores que dificultam o processo de elaboração do planejamento de TI?".

%O objetivo geral desta pesquisa é identificar empiricamente, através de uma teoria fundamentada em dados, os fatores que dificultam ou impedem a elaboração do planejamento de TI em instituições públicas federais brasileiras. Assim, (hipótese) diante da compreensão destes fatores, seria possível propor ações práticas para a melhoria deste cenário.


%\section{Considerações finais}

Em um cenário de baixo nível de maturidade em governança de TI nas instituições públicas federais do Brasil, conforme apontam relatórios dos órgãos de controle, esta pesquisa abordou o problema do planejamento de TI nestas instituições. Compreender as razões que levam às dificuldades de planejamento de TI neste cenário compõe o objetivo geral desta pesquisa, que é sintetizado da seguinte forma: identificar empiricamente, através de uma teoria fundamentada em dados, os fatores que dificultam ou impedem a elaboração do planejamento de TI em instituições públicas federais brasileiras. Assim, diante da compreensão destes fatores, seria possível propor ações práticas para a melhoria deste cenário.

Os objetivos específicos atingidos nesta pesquisa são:
\begin{enumerate}
\item Elaborar teoria fundamentada em dados sobre o problema de planejamento de TI apresentando causa, fenômeno e consequência, conforme apresentado no Capítulo 4, Seção \ref{secao:resultado_teorias};
\item Enfatizar as melhores práticas de planejamento estratégico de TI aplicáveis à teoria obtida, conforme exposto no Capítulo 5, Seção \ref{secao:mp_finais}. 
\end{enumerate}

Para que fosse possível elaborar uma teoria com fundamentação na realidade foi realizada a coleta de dados com participantes da elaboração de planejamento de TI de diversas instituições federais. A \textit{Grounded Theory}, como método científico, cumpriu seu papel permitindo que o objetivo geral da pesquisa fosse atingido: identificou-se empiricamente, através de teoria fundamentada em dados, os fatores que dificultam e impedem a elaboração do planejamento de TI nas organizações públicas. Além disso, foi possível apontar um conjunto de melhores práticas para a melhoria dos problemas encontrados na elaboração do planejamento de TI.

Os objetivos específicos, subprodutos do objetivo principal, também foram cumpridos. Foi possível obter duas teorias fundamentadas nos dados coletados, uma para cada situação-problema. O segundo objetivo específico foi possível através do banco de melhores práticas em planejamento estratégico de SI/TI do modelo de maturidade MMPE-SI/TI (Gov). A partir dos fatores componentes das teorias resultantes da pesquisa foi possível descobrir as melhores práticas de planejamento estratégico relacionadas à cada elemento das teorias. 
%Os resultados dos objetivos específicos foram apresentados na \autoref{secao:resultado_teorias} e \autoref{secao:melhores_praticas}. 

Apesar de ter seguido à rigor o método \textit{Grounded Theory}, que pela sua natureza dispensa validações, este trabalho também apresentou uma avaliação das teorias fundamentadas nos dados. Foi proposto um questionário onde os mesmos participantes da coleta de dados pudessem avaliar o grau de aderência da teoria à realidade observada por eles. Os resultados da avaliação, apresentados na \autoref{secao:avaliacao_resultados}, são considerados satisfatórios para validar o grau de fundamentação nos dados das teorias apresentadas.

Além de atingir os objetivos, esta pesquisa originou algumas lições aprendidas ao autor. Com relação ao método, conclui-se que há um risco considerável em aplicar \textit{Grounded Theory} em dados coletados por questionário. Este risco ocorre pois, para utilizar GT, é preciso uma massa de dados considerável e isto nem sempre é possível em questionários. As perguntas devem ser construídas de forma a induzir o participante à respondê-las de forma detalhada, com número razoável de palavras. Pressupõe-se que em entrevistas presenciais este risco pode ser reduzido.

Outra lição aprendida nesta pesquisa consiste na disciplina exigida na aplicação da \textit{Grounded Theory}. Registrar anotações sobre os pensamentos, proposições e ações que o pesquisador realiza durante o processo é essencial para manter a rastreabilidade dos conceitos que originaram a teoria. Armazenar as notas de cada fase da GT permitiu evoluir a teoria sem perder coerência e fundamentação.

Esta pesquisa mostrou que, apesar da grande área da Ciência da Computação ser em sua essência uma área das ciências exatas, existem problemas complexos de ordem subjetiva que podem ser resolvidos com métodos qualitativos e conceitos de áreas consideradas ciências humanas, como a administração e a sociologia.



\section{Limitações e Ameaças}
%: “Se você não for o maior crítico de seu próprio trabalho, outra pessoa será”.

Apesar desta pesquisa ter cumprido seus objetivos, há algumas limitações neste trabalho que devem ser consideradas:
\begin{itemize}
\item A coleta de dados desta pesquisa se restringiu à organizações públicas federais de um mesmo nicho, instituições de ensino, pesquisa e extensão;
\item Os respondentes da coleta de dados atuaram ou atuam na área de TI das instituições alvo. Este fato pode contribuir para um viés nos resultados, apesar de não torná-los inválidos pois, ainda assim, trata-se de uma perspectiva real e deve ser refletida nos dados;
\item O autor teve neste trabalho o seu primeiro contato com o método \textit{Grounded Theory}. Apesar de adquirir o conhecimento do método suficiente para utilizá-lo, observa-se que a inexperiência com o método é um limitante;
\item O questionário de avaliação dos resultados não foi respondido pela totalidade dos participantes da coleta de dados;
\end{itemize}

\section{Contribuições}

As principais contribuições desta dissertação são apresentadas a seguir:
\begin{itemize}
\item Desenvolveu-se uma teoria substancial, fundamentada em dados, que permite compreender os fatores que levam às instituições à não realizarem o planejamento de TI;
\item Desenvolveu-se uma teoria substancial, fundamentada em dados, que permite compreender os fatores que dificultam a elaboração do planejamento de TI e que comprometem a qualidade do plano;
\item Realizou-se a seleção das melhores práticas de planejamento estratégico de TI que podem minimizar os fatores que contribuem para o problema do planejamento de TI nas instituições públicas pesquisadas. Esta contribuição permite inferir que os fatores que restringem a elaboração do PDTI são problemas já vivenciados por outras organizações e possuem solução através das melhores práticas de planejamento;
\item Apresentou-se trabalhos recentes relacionados aos fatores condicionantes do planejamento de TI no setor público brasileiro;
\item Aplicou-se, pela primeira vez, o método \textit{Grounded Theory} com o intuito de compreender o problema de planejamento de TI nas instituições públicas federais brasileiras. Todo o processo de aplicação do método foi documentado e apresentado nesta dissertação, contribuindo para possíveis pesquisas que desejam utilizar este método em cenários semelhantes.
\end{itemize}

Esta pesquisa também apresentou uma avaliação das teorias descobertas, conforme seção \ref{secao:avaliacao_resultados}. Ressalta-se que apesar de não atingir completamente a amostragem da coleta de dados, a avaliação dos resultados se mostrou satisfatória comprovando a fundamentação das teorias. Não houve nenhuma discordância na avaliação das teorias e a teoria 1 apresentou 100\% de aderência, enquanto a teoria 2 apresentou 79,3\% de aderência. Portanto, os respondentes se viram representados nas teorias elaboradas nesta pesquisa.

Por fim, o legado deste trabalho consiste não somente nos resultados da pesquisa em si e nas contribuições científicas. Mas também no fornecimento dos insumos necessários para a construção de um plano de ações efetivas que possam atacar as origens do problema da falta de planejamento de TI nas instituições públicas. Contribuindo, desta forma, para a melhoria dos serviços prestados ao cidadão e para o melhor emprego dos orçamentos destinados às áreas de TI dos entes públicos.

\section{Perspectivas futuras}

Embora este trabalho tenha cumprido seus objetivos, compreende-se que há oportunidades de pesquisa que podem ser exploradas em trabalhos futuros:

\begin{itemize}
\item Aplicar uma pesquisa semelhante a esta, porém com gestores e membros das áreas de negócio com o objetivo de obter uma teoria fundamentada em dados sob a perspectiva de participantes que não atuam na área de TI das instituições. Desta forma, sob o ponto de vista das áreas de negócio, poderiam ser descobertos novos fatores que contribuem para o problema do processo de elaboração do planejamento de TI;
\item Avaliar as teorias obtidas nesta pesquisa no contexto de outras instituições públicas, com o intuito de medir a aderência das teorias em órgãos de outras áreas. Uma vez medida a aderência das teorias em outras instituições seria possível descobrir diferenças organizacionais entre elas que influenciam positiva ou negativamente no processo de planejamento de TI;
\item Desenvolver um método de aplicação das melhores práticas de planejamento de TI: elaboração um plano de ações operacionais a serem executadas nas instituições alvo desta pesquisa. Uma ferramenta que apoie operacionalmente as instituições a atingirem os processos de planejamento selecionados nesta pesquisa e que forneça insumos para medir a evolução da execução de cada processo facilitaria a melhoria da maturidade em governança de TI nestas instituições;
\item Realizar estudos de caso para avaliar as possíveis melhorias resultantes da aplicação das melhores práticas de planejamento de TI. Avaliar empiricamente a evolução das instituições que se propõem a aplicar as melhores práticas pode trazer grande contribuição para a consolidação das MP como ferramentas de melhoria do planejamento de TI no setor público;
\item Desenvolver uma pesquisa de natureza semelhante, com teoria fundamentada em dados, acerca dos problemas de elaboração de planejamento de TI em instituições privadas. Uma comparação entre os fatores influenciadores no planejamento de TI em instituições públicas e privadas poderia ampliar o conhecimento acerca destes dois setores, além de realçar semelhanças e diferenças;
\item Aplicar o método \textit{Grounded Theory} com o objetivo de descobrir fatores influenciadores ou que dificultam a execução e o monitoramento e controle dos planos de TI. Desta forma, juntamente com a presente pesquisa, seria possível uma compreensão das dificuldades nas principais fases do planejamento de TI.
\end{itemize}